\documentclass[a4paper,10pt]{article}
\usepackage{url,a4wide}
\setlength{\parindent}{0pt}
\setlength{\parskip}{1ex}
\setlength{\itemsep}{0pt}
\pagestyle{empty}
\renewcommand{\theenumi}{\alph{enumi}}
\begin{document}

\centerline{\LARGE\bf UNIB20005: Language and Computation}\vspace{2ex}

\centerline{\large\bf Project 3: Grammar Development}\vspace{2ex}

A context-free grammar is a set of rewriting rules, or productions, that models
our intuitions about grammatical sentences.  Starting with the grammar
in \url{http://langtech.github.com/lac/projects/english.fcfg},
you will extend and test the grammar, and discuss your findings.

You may do this project individually, or in a team of two.
(Team submissions should have wider coverage and more test cases,
as specified in parentheses below.)

A grammar \emph{test suite} is a collection of sentences that can be used to
check that a grammar only accepts grammatical sentences.  It contains a mixture of
grammatical and ungrammatical sentences.  A problem exists in the grammar if it
rejects a grammatical sentence, or if it accepts an ungrammatical sentence.
Remember that a sentence does not have to be meaningful in order to be
grammatical (i.e.\ syntactic well-formedness is not the same as semantic interpretability).
Obtain the test code from \url{http://langtech.github.com/lac/projects/gde.py}. 

\begin{enumerate}
\item Create a test suite for the above grammar, in a file
\url{sentences.txt}, containing at least 15 (25) grammatical sentences, and at least
15 (25) ungrammatical sentences, one per line.  Use all lowercase, avoid
punctuation, and separate each word with whitespace.   Use words that
are covered in the grammar.
Mark ungrammatical sentences with an asterisk
at the start of the line.  For example:

\begin{verbatim}
Jody claimed several cars disappeared
*Jody walked several cars disappeared
\end{verbatim}

\item Extend the grammar to support another kind of verb subcategorization,
such as \emph{put} (which requires an object noun phrase and a locative
prepositional phrase), or \emph{give} (which requires two noun phrases, or a
single noun phrase followed by a prepositional phrase headed by \emph{to}).
Extend the test suite with at least 10 more grammatical sentences, and 10 more
ungrammatical sentences. (Teams should do this twice, for two different verb types.)

\item Now extend the grammar with several more productions, by adding support
for one (two) more syntactic construction(s), such as
questions, relative clauses, cleft sentences, adverbial clauses
(see the \emph{SIL Glossary of Linguistic Terms} for explanations
and examples).  Take special care with your use of features.
Use the lectures, and the syntax handout
as a further source of ideas about possible syntactic constructions.  Extend the
test suite with at least 15 (30) more grammatical sentences, and 15 (30) more ungrammatical sentences.

\end{enumerate}

Discuss your work in a plain text file named \verb|report.pdf| or \verb|report.txt| ($\sim 400 (750)$ words).
Identify the team members in each file submitted.
Your work will be assessed for correctness, clarity and insight. 
The report must be original.
Your submission is worth 10\% of the marks for this subject.

\textbf{Note.}  The project has made simplifying assumptions
relative to current approaches in ``grammar engineering'',
which check that the expected parse trees and semantic
representations were produced.

Please email your files (grammar, test sentences, report) with subject line
\textit{L\&C Project 3}
to Steven Bird at \texttt{sbird@unimelb.edu.au},
copying any team members.  All submissions will be acknowledged.
If you do not receive an acknowledgement within three days of the submission deadline, please resend.

Submit your work by the end of week 12 (10pm on Friday 26 October).

\end{document}
