\documentclass[a4paper,10pt]{article}
\usepackage{url,a4wide}
\setlength{\parindent}{0pt}
\setlength{\parskip}{1ex}
\setlength{\itemsep}{0pt}
\pagestyle{empty}
\renewcommand{\theenumi}{\alph{enumi}}
\begin{document}

\centerline{\LARGE\bf UNIB20005: Language and Computation}\vspace{2ex}

\centerline{\large\bf Project 1: Processing Low Quality Text}\vspace{2ex}
\vfil

Some text and speech input applications are able to insert punctuation and guess the correct casing of text.

Your task is to write a program to simulate this behaviour. For example, it will take
the input:
\begin{blockquote}
we collect and preserve the abc radio and television recordings that have documented the cultural life of australians since the first radio broadcast in 1932.
\end{blockquote}

And produce the following output:

\begin{blockquote}
We collect and preserve the ABC Radio and Television recordings that have documented the cultural life of Australians since the first Radio broadcast in 1932.
\end{blockquote}









Submit report approximately five pages long (1500 words), in PDF format.
Name the file \texttt{uname.pdf} where \textit{uname} is your university
username (all lowercase, no punctuation).

Your report should cover the following aspects (and you should use these
section headings):
\begin{description}\setlength{\itemsep}{0pt}\setlength{\parskip}{0pt}
\item \textbf{1. Motivation}: why would someone need this technology?  are there
different versions to meet different needs?
\item \textbf{2. Operation}: which actual system did you use?
how does it work?  what kinds of knowledge
does it require?  (include technical detail)
\item \textbf{3. Examples}: give examples of actual system input and output;
try to illustrate interesting aspects of the system
\item \textbf{4. Evaluation}: how effective is the system? what are the strengths
and weaknesses? what kinds of mistakes does it make and why?
\item \textbf{5. Impacts}: if this technology was widely adopted, what effects
might it have on language and society?
\end{description}

Your work will be assessed for correctness, clarity, and insight.
The report must be original work.  Literal quotations must be clearly
marked, and must be accompanied by a citation.
All sources should be cited in a references section at the end.
The report is worth 10\% of the total marks
for this subject.

Some example technologies and available systems are listed below.
(You are welcome to choose other language technologies; please check
with the coordinator if you are not sure the technology is relevant.)
\begin{description}\setlength{\itemsep}{0pt}\setlength{\parskip}{0pt}
\item[Question answering:]\hfil\\
\url{http://en.wikipedia.org/wiki/Question_answering}\\
\url{http://www.answerbus.com/}

\item[Automatic summarization:]\hfil\\
\url{http://en.wikipedia.org/wiki/Automatic_summarization}\\
\url{http://newsblaster.cs.columbia.edu/}

\item[Machine translation:]\hfil\\
\url{http://en.wikipedia.org/wiki/Machine_translation}\\
\url{http://translate.google.com/}

\item[Dialogue systems:]\hfil\\
\url{http://en.wikipedia.org/wiki/Dialog_system}\\
\url{http://www.chatbots.org/}

\item[Speech recognition:]\hfil\\
\url{http://en.wikipedia.org/wiki/Speech_recognition}\\
\url{http://www.speechapi.com/}

\item[Grammar checker:]\hfil\\
\url{http://en.wikipedia.org/wiki/Grammar_checker}\\
\url{http://www.spellchecker.net/grammar/}

\end{description}

Submit your work via the LMS by the end of week 4 (10pm on Friday 17 August).

\end{document}


