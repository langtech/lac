\documentclass[a4paper,10pt]{article}
\usepackage{url,a4wide}
\setlength{\parindent}{0pt}
\setlength{\parskip}{1ex}
\setlength{\itemsep}{0pt}
\pagestyle{empty}
\begin{document}

\centerline{\LARGE\bf UNIB20005: Language and Computation}\vspace{2ex}

\centerline{\large\bf Project 2: Grammars, Parsing and Semantics}\vspace{2ex}

A context-free grammar can be used to model our intuitions about grammatical sentences.
It can be enriched with semantic representations, a compositional theory of meaning.
In this project you will develop a small grammar, and use it with a parser to
process sentences and produce appropriate semantic representations.

You are encouraged to do this project in teams of two or three.

Use the following grammar as a starting point:\\
\url{http://lp20.org/lac/projects/grammar.fcfg}

And use the following code to document and test your grammar:\\
\url{http://lp20.org/lac/projects/grammar_test.py}

\paragraph{Tasks.}
You are to extend the grammar in the following ways:

\begin{enumerate}
\item proper names
\item number agreement
\item transitive and ditransitive verbs
\item one other construction of your choosing, e.g. prepositional phrases, relative clauses, sentential complements, questions, adverbial phrases (see the \emph{SIL Glossary of Linguistic Terms} for explanations
and examples).
\end{enumerate}

Discuss your work inside the Python file (approx 400 words).
Identify the team members at the top of this file.
Your work will be assessed for correctness, clarity and insight. 
The work must be original.
Your submission is worth 10\% of the marks for this subject.

\paragraph{Submission:}
Your project is due at the end of week ten (10pm, Friday 11 October). Please submit it by email to
\texttt{sbird@unimelb.edu.au, nj@unimelb.edu.au}, copying both project members. Please use the subject line: \texttt{L\&C Project 2}.
All submissions will be acknowledged, so if you do not receive acknowledgement, please contact staff directly.
Please retain the existing two filenames unchanged, and attach both to the email.

\end{document}
