\documentclass[a4paper,10pt]{article}
\usepackage{url,a4wide}
\setlength{\parindent}{0pt}
\setlength{\parskip}{1ex}
\setlength{\itemsep}{0pt}
\pagestyle{empty}
\renewcommand{\theenumi}{\alph{enumi}}
\begin{document}

\centerline{\LARGE\bf UNIB20005: Language and Computation}\vspace{2ex}

\centerline{\large\bf Project 1: Language Technologies}\vspace{2ex}
\vfil

A variety of useful ``language technologies'' have been developed, with the
goal of improving the usability of devices and services, and improving access
to stored information.  In this project, you will research a language
technology of your choice.  You will need to interact with an actual working
system, and report on your experience.

Submit report approximately five pages long (1500 words), in PDF format.
Name the file \texttt{uname.pdf} where \textit{uname} is your university
username (all lowercase, no punctuation).

Your report should cover the following aspects (and you should use these
section headings):
\begin{description}\setlength{\itemsep}{0pt}\setlength{\parskip}{0pt}
\item \textbf{1. Motivation}: why would someone need this technology?  are there
different versions to meet different needs?
\item \textbf{2. Operation}: which actual system did you use?
how does it work?  what kinds of knowledge
does it require?  (include technical detail)
\item \textbf{3. Examples}: give examples of actual system input and output;
try to illustrate interesting aspects of the system
\item \textbf{4. Evaluation}: how effective is the system? what are the strengths
and weaknesses? what kinds of mistakes does it make and why?
\item \textbf{5. Impacts}: if this technology was widely adopted, what effects
might it have on language and society?
\end{description}

Your work will be assessed for correctness, clarity, and insight.
The report must be original work.  Literal quotations must be clearly
marked, and must be accompanied by a citation.
All sources should be cited in a references section at the end.
The report is worth 10\% of the total marks
for this subject.

Some example technologies and available systems are listed below.
(You are welcome to choose other language technologies; please check
with the coordinator if you are not sure the technology is relevant.)
\begin{description}\setlength{\itemsep}{0pt}\setlength{\parskip}{0pt}
\item[Question answering:]\hfil\\
\url{http://en.wikipedia.org/wiki/Question_answering}\\
\url{http://www.answerbus.com/}

\item[Automatic summarization:]\hfil\\
\url{http://en.wikipedia.org/wiki/Automatic_summarization}\\
\url{http://newsblaster.cs.columbia.edu/}

\item[Machine translation:]\hfil\\
\url{http://en.wikipedia.org/wiki/Machine_translation}\\
\url{http://translate.google.com/}

\item[Dialogue systems:]\hfil\\
\url{http://en.wikipedia.org/wiki/Dialog_system}\\
\url{http://www.chatbots.org/}

\item[Speech recognition:]\hfil\\
\url{http://en.wikipedia.org/wiki/Speech_recognition}\\
\url{http://www.speechapi.com/}

\item[Grammar checker:]\hfil\\
\url{http://en.wikipedia.org/wiki/Grammar_checker}\\
\url{http://www.spellchecker.net/grammar/}

\end{description}

Submit your work via the LMS by the end of week 4 (10pm on Friday 17 August).

\end{document}


